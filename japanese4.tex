\documentclass[lualatex, ja=standard, fontsize=10.5bp]{jlreq}

\usepackage[top=35mm, bottom=35mm, left=30mm, right=30mm]{geometry}
\usepackage{framed}
\usepackage{enumitem}

\newcommand{\kou}{売り子} %登場人物①
\newcommand{\otsu}{D.K.} %登場人物②

\begin{document}
\begin{minipage}{0.7 \hsize}
    \begin{flushleft}
        {\Large 国語Ⅳ }

        {TR.〿 バナナの展示即売会で\otsu の質問に答える}
    \end{flushleft}
\end{minipage}
\begin{minipage}{0.3 \hsize}
    \begin{flushright}
        {2022年〿月〿日 〿限}

        {学籍番号 名前}
    \end{flushright}
\end{minipage}

\-

\paragraph*{本発表の構成}
テーマ理解 -> 問題点とその解決法 -> 各設問 -> まとめ・補足

\-

\paragraph*{問11}
\kou くんの答えかたの問題点について
\begin{enumerate}[label=(\arabic*)\,]
    \item 「無農薬栽培」のメリットはゴリラにはわかりにくいかもしれない。
    \item ゴリラにもわかりやすい話し方ではない
\end{enumerate}

\-

\paragraph*{問12}
お客さんの質問に対し、\kou くんが上手に応答する会話例
\begin{framed}
    \begin{itemize}
        \setlength{\leftskip}{0.5cm} %箇条書きのタイトルが枠線にかぶるのでこいつでずらす
        \item[\kou] 「\underline{このように...}」
        \item[\otsu] 「ウホッ」  
        \item[\kou] 「であるからして...」
        \item[\otsu] 「オホホ」   
    \end{itemize}
\end{framed}

\end{document}